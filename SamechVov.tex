
% These lines tell TeXShop to typeset with xelatex, and to open and 
% save the source with Unicode encoding.

%!TEX TS-program = xelatex
%!TEX encoding = UTF-8 Unicode

\documentclass[10pt]{book}
\usepackage{anyfontsize}
\usepackage{geometry} 
\geometry{letterpaper}

\usepackage{titlesec}
\titleformat{\section}[block]{\bfseries\filcenter}{}{1em}{}

% \usepackage{blindtext}
\usepackage{lettrine}
\setlength{\parskip}{1em}

\usepackage{graphicx}
\usepackage{amssymb}

\usepackage{xltxtra,polyglossia,babel}
% xltxtra automatically loads fontspec and xunicode

\setmainlanguage{hebrew}
\newfontfamily\hebrewfont[Scale=1.4,BoldFont={Miriam CLM}]{Frank Ruehl CLM}

\setotherlanguages{english}
\setromanfont[Mapping=tex-text]{TeX Gyre Pagella}

\title{
יום טוב של ראש השנה
\vskip 0.1em
{\fontsize{50}{60}\selectfont
תרס״ו%
}
\vskip 0.1em
\begin{english}
Yom Tov shel Rosh Hashanah -- 5666%
\end{english}}
% \subtitle
\author{
מאת
\\
\ \\
כ״ק אדמו״ר%
\\
אור עולם נזר ישראל ותפארתו כקש״ת%
\ \\
\begin{LARGE}מוהר״ר שלום דובער\end{LARGE}%
\ \\
\ \\
זצוקללה״ה נבג״ם זי״ע מליובאוויטש%
\ \\
\ \\
\ \\
by\\
\ \\
His Honorable Holiness,\ Admor,\\
Our Teacher the Rebbe,\\
\ \\
\begin{LARGE}Rabbi Shalom Dovber\end{LARGE}\\
\begin{LARGE}Schneersohn\end{LARGE}\\
of Lubavitch
}
\date{}

\begin{document}
\maketitle

\section*{
בס״ד, ליל ב׳ דר״ה, רס״ו
\\
\begin{english}
With the help of Heaven, the second night of Rosh Hashanah, 5666
\end{english}
}
\leftskip=0pt plus-.5fil
\rightskip=0pt plus.5fil
\parfillskip=0pt plus1fil
\lettrine[findent=0.75em, nindent=0em, loversize=-.2, lraise=.45]{יום}
 טוב של ר״ה שחל להיות בשבת במקדש היו תוקעין אבל לא המדינה, ומפרש בגמ׳ התעם משום גזירה דרבה, דאמר רבה הכל חייבין בתקיעת שופר ואין הכל בקיאין בתקיעת שופר, גזירה שמא יטלנו בידו וילך אצל הבקי ללמוד ויעבירנו ד״א ברה״ר. ולכאורה אינו מובן, מה ראו חז״ל לעקור מ״ע דאורייתא משום חשש גזירה בעלמא, והלא החשש הוא להדיוטים וקלי הדעת, ואיך מנעו המצוה לגמרי מכמה צדיקים גדולים וטובים ובפרט מצות שופר שהיא מצוה רמה ונשאה מאד, וכידוע דכל המצות שבחדש זה הם מצות כלליות, וכמ״ש במד״ר ע״פ ובחדש השביעי שמושבע בכל, שיש בו הרבה מצות, וחלוקים המה מהמצות דכל השנה, שהן מצות כלליות, ובפרט מצות שופר שמעורר וממשיך פנימיות ועצמות אוא״ס כו׳.

\begin{english}
    \leftskip=0pt \rightskip=0pt
\noindent \lettrine{I}{f} the holiday of Rosh Hashanah fell on Shabbat, [the shofar] would be sounded in the Temple, but not [throughout the rest of] the country. It is explained in the Gemara that this is because of the decree of Rabbah, who said, ``All are required [to hear] the sounding of the shofar, but not all are knowledgeable in the sounding of the shofar. [Let there be] a decree [forbidding its sounding on the Shabbat], lest one take [the shofar] in his hand and go to one knowledgeable in order to learn [how to sound it], and so [transgress the Shabbat by] traversing four \emph{amot} in a \emph{reshut harabim.}''

    But at first sight, this is not understood. Why did \emph{chazal} see [fit] to uproot a positive mitzvah of the Torah, based only on a possibility of [violating] a decree? [Furthermore,] the possibility is [only] for regular folk and the easily-distracted. [If so,] how could they completely withhold the mitzvah from so many great and good \emph{tzadikim}? and especially such a lofty and exalted mitzvah as the mitzvah of shofar! As is known, all the mitzvot of this month [of Tishrei] are all-encompassing mitzvot, as it says in \emph{Midrash Rabbah} regarding the verse ``And in the seventh \emph{[shevi'i]} month...'' [it is so termed] because it is satiated \emph{[mus'va]} with everything, since it has many mitzvot. [Futhermore,] these [mitzvot] are different from the mitzvot of the rest of the year, because they are all-encompassing mitzvot. This is especially so of the mitzvah of shofar, which awakens and draws down the inner aspect and essence of the \emph{Or Ein Sof}....

\end{english}

\leftskip=0pt plus-.5fil
\rightskip=0pt plus.5fil
\parfillskip=0pt plus1fil
\lettrine[findent=0.75em, nindent=0em, loversize=-0.5, lraise=0.67]{וידוע}
דמצות שופר בכוונתה הרוחניות הו״ע התשובה, וזהו״ע התקיעה הראשונה שהוא בחי׳ צעקה בקול פשוט, שזהו מצד ההתעוררות והתפעלות פנימי שבנפש שבא בהתגלות בקול פשות דוקא. דהנה יש בחי׳ התפעלות פנימי דנפש בעבודה דרעותא דלבה ג״כ, והיא בחשאי דוקא, וכמו עבודת הכוהנים, דכהנא בעובדא וברעו״ד שהיא בחשאי כו׳ כידוע, וכאן הוא בצעקת הקול דוקא, והו״ע התשובה, כי שניהם הם בבחי׳ התפעלות פנימי דנפש. אמנם חלוקים המה בענינם, דברעו״ד סיבת ההתפעלות היא מצד הקירוב, והיינו מצד הרגש האוא״ס שמרגיש בנפשו, והוא כשמתבונן בבחי׳ עצמות אוא״ס שמופלא ומרומם בעולמות שאינו בא בבחי׳ גילוי בנפשו (מפני שאין נשמתו כלי לזה שיאיר האור הזה בפנימיות בגילוי בנפש), עי״ז נעשה הרצוא להכלל באוא״ס כו׳, כמשי״ת, ועצם ענין ברצוא ג״כ ענינה הוא בחי׳ קירוב מה שרוצה להכלל כו׳, משע״כ בתשובה הרי אין זה מצד הרגש אוא״ס בנפשו, כ״א מצד הריחוק דוקא, שצר לו מאד מה שנתרחק. ויש בזה ג״כ הרגש האלקות, אך זהו מה שרע ומר עזבו את הוי׳ כו׳, אבל לא שמרגיש אוא״ס ההפלאה והעלוי ומזה הוא התפעלות נפשו, כ״א מה שנוגע מאד בעומק לבבו הריחוק, מזה נעשה ההתפעלות כו׳, ויש בזה ג״כ ההמשכה (דער ציען זעך) והרצוא לעלקות, אבל ענינו אינו ענין הקירוב, דהיינו שיהי׳ בזה ג״כ המבוקש לכלל כו׳, כ״א מה שרוצה להיות מקורב לאלקות, היינו שלא יהי׳ המקום הריחוק כ״א בבחי׳ הקירוב, והוא רק ענין הנתינה והמסירה שמוסר א״ע לאלקות ולקיים רצונו כו׳.

\begin{english}
    \leftskip=0pt \rightskip=0pt
\noindent \lettrine{I}{t} is known that the mitzvah of the shofar is, in its spiritual intention, about \emph{teshuvah.} [Indeed,] this is the idea of the first blast \emph{[teki'ah]} which specifically embodies the quality of a simple cry. This parallels the inner awakening and feeling within the \emph{nefesh}, which is revealed particularly in a simple cry. [In a subtle counterpoint to this,] there is also a quality of inner feeling in the \emph{nefesh} in the midst of the service of arousal of the heart. This [quality] must be silent, similar to the service of the priests [in the Temple,] as ``the priests' [duty] was in service and in arousal of the heart,'' which was in silence, etc. as is known. But here, [we are discussing] specifically the outburst of a cry, and the idea here is of \emph{teshuvah,} although both involve the quality of inner feeling in the \emph{nefesh.} 

Nevertheless, they are qualitatively different. 

On one hand, with the arousal of the heart, the emotion is because of closeness [to \mbox{G-d}], i.e. because of the sensation of the \emph{Or Ein Sof} that one feels in one's \emph{nefesh.} This comes when one meditates on the quality of the Essence of the \emph{Or Ein Sof}, which is wondrous and exalted in worlds that one cannot perceive consciously in one's \emph{nefesh} (because one's \emph{nefesh} is not a [fitting] vessel for this, that this Light should shine internally, in revelation in the \emph{nefesh}). Through this [meditation], there will be made a desire to be included [and to be nullified] in the \emph{Or Ein Sof}, etc., as will be explained. The essential idea in [this] desire is also the quality of closeness [to \mbox{G-d}], in that one wishes to be included....

[On the other hand,] such is not so of \emph{teshuvah.} This [feeling] is not because of a [palpable] sensation of the [infinite] \emph{Or Ein Sof} in one's \emph{nefesh,} but rather particularly because of distance, as it pains one exceedingly that one has become distant [from \mbox{G-d}]. In this also there is a sensation of \mbox{G-dliness}, but this is [the feeling of] how evil and bitter [it is, that one] abandoned \emph{Havayeh....} It is not that one perceives the wondrous and exalted \emph{Or Ein Sof,} and because of this his \emph{nefesh} is moved. Rather, one is sorely stricken in the depths of one's heart from the distance, and from that, a feeling is made.... And in this, too, there is a draught ([in Yiddish,] \emph{der tziehen zech}, [roughly, drawing oneself to a higher plane]) and a desire for \mbox{G-dliness.} Yet, the idea here is not an idea of closeness, i.e. that there is also a seeking to be included [within the \emph{Or Ein Sof}], etc. Rather, what one desires [here] in being close to \mbox{G-dliness} is that there should no longer be a state of distance, but rather closeness. This is but a matter of dedicating and giving oneself over, of giving oneself up to \mbox{G-dliness} and fulfilling His Will....

\end{english}
\end{document}  

